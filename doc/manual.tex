%% LyX 2.0.6 created this file.  For more info, see http://www.lyx.org/.
%% Do not edit unless you really know what you are doing.
\documentclass[oneside,english]{book}
\usepackage[T1]{fontenc}
\usepackage[latin9]{inputenc}
\usepackage{listings}
\setcounter{secnumdepth}{3}
\setcounter{tocdepth}{3}
\usepackage{textcomp}
\usepackage{babel}
\begin{document}

\title{Animation Library}


\author{�2013 Majenko Technologies}
\maketitle
\begin{quotation}
The Animation library is a simple parallel processing script engine
for manipulating large number of LEDs through a number of chained
ULK devices.
\end{quotation}

\chapter{Animation Concepts}

The Animation library runs a number of \textit{programs} in parallel,
where each \textit{program} is a list of special instructions and
their associated parameters. Instructions are executed in blocks separated
by \textit{delay} instructions. When a \textit{delay} instruction
is encountered the program is released and the next \textit{program}
is executed. Each animation instance has its own context and no animation
can affect another animation.

\textit{Programs} are stored as arrays of 8-bit unsigned characters
(typically \textit{uint8\_t}, \textit{byte} or \textit{unsigned char})
and must be terminated by one of the \textit{terminating} instructions,
such as LOOP or END.

Simple counted repeats are supported, as well as infinite repeats
with an external interrupt (known as a \textit{nudge}) to force loop
termination.

Operations can be grouped into three broad categories: \textit{control},
\textit{LED} and \textit{flow}. \textit{Control} instructions perform
such facilities as LED group definitions, \textit{LED} instructions
directly manipulate the state of LEDs and groups of LEDs and \textit{flow}
instructions define how the program proceeds.

Each instruction is stored internally as an 8-bit unsigned value,
or \textit{op-code}, and the library provides a number of symbolic
names for these \textit{op-codes} to enable simpler and more readable
programming.


\chapter{Animation Instructions}


\section{Control Instructions}


\subsection{GROUP (op-code 1)}

The GROUP command defines a new (or replaces an existing) LED group
definition. A group of LEDs can all be manupilated together with LED
instructions through the use of a group number instead of an LED number.

The GROUP command takes a minimum of 3 parameters, but can take many
more. The first parameter is the number of the group to define or
redefine. Groups are numbered between 200 and 255 inclusive. The second
parameter is the number of LEDs in the group. This is then followed
by that number of parameters, one for each LED. For instance, to define
group 200 with 3 LEDs (6, 7 and 8) in it, the command structure would
be:

\texttt{Animation::GROUP, 200, 3, 6, 7, 8}


\section{LED Commands}


\subsection{SET (op-code 2)}

The set command is used to directly set the brightness of either a
single LED or a group of LEDs defined with the GROUP command. The
command takes 2 parameters - the number of the LED or GROUP to change,
and the value to change it to. For instance, to set the LED number
7 to 50\% brightness the command would be:

\texttt{Animation::SET, 7, 128}

Or to set the GROUP 200 to full brightness:

\texttt{Animation::SET, 200, 255}


\subsection{FADE (op-code 3)}

The FADE command, like the SET command, changes the brightness of
an LED or group of LEDs. Unlike the SET command, however, the change
isn't immediate. Instead the LED or GROUP fades gradually between
the current brightness and the desired target brightness. The rate
of the change is specified as a delay in milliseconds between each
step of brightness.

The command takes 3 paramaters. The first is the number of the LED
or GROUP to change. The second is the target brightness value (0-255)
and the third is the number of milliseconds delay per step of the
fade.

Example - fade LED 4 up to full brightness with a delay of 10ms per
step:

\texttt{Animation::FADE, 4, 255, 10}


\section{Flow Commands}


\subsection{DELAY (op-code 4)}

The DELAY command pauses the program for the specified number of milliseconds.
The command takes two parameters, which are the high and low bytes
of a 16-bit value, most significant byte first. For instance, a 1
second delay, which is 1000 milliseconds, would be represented as
the two bytes 3 and 232 (3 {*} 256 + 232 = 1000):

\texttt{Animation::DELAY, 3, 232}

A preprocessor macro PAIR is provided to perform this splitting for
you:

\texttt{Animation::DELAY, PAIR(1000)}


\subsection{WAITEQ (op-code 5)}

The WAITEQ command delays the program until any FADE instructions
have finished processing (it stands for WAIT until EQual). It takes
no parameters:

\texttt{Animation::WAITEQ}


\subsection{RDELAY (op-code 6)}

The RDELAY command performs a random length delay. Like the DELAY
command the delay is measured in milliseconds, and is provided as
a pair of bytes representing a 16-bit value. Two values are provided
as parameters representing the minimum and maximum delay times to
use. Again, the PAIR macro can be used. For example, for a random
delay between 100 and 1000 milliseconds:

\texttt{Animation::RDELAY, PAIR(100), PAIR(1000)}


\subsection{REPEAT (op-code 7)}

This command repeats the following block of code, up to the first
LOOP command found, a specified number of times, up to 255. Only one
parameter is provided, which is the number of iterations to perform:

\texttt{Animation::REPEAT, 10}

\texttt{... code ...}

\texttt{Animation::LOOP}


\subsection{FOREVER (op-code 8)}

FOREVER acts like the REPEAT command, however a number of iterations
is not specified. The following block repeats indefinitely until a
\textit{nudge} is sent to the animation at which point the loop finishes
and the program continues:

\texttt{Animation::FOREVER}

\texttt{... code ...}

\texttt{Animation::LOOP}


\subsection{LOOP (op-code 254)}

This is one of the instructions classed as a terminating instruction
and as such is one of the few instructions which may be used as the
final instruction of a program.

The LOOP command returns the program either back to the previous REPEAT
or FOREVER command (if one has been used) or to the start of the program.
If a REPEAT or FOREVER has not been previously issued the LOOP will
always loop the program regardless of any nudges:

\texttt{Animation::LOOP}


\subsection{END (op-code 255)}

The END command terminates the program. Any LEDs affected by the program
remain at their final settings until changed by another animation.
The animation may be manually restarted by the user:

\texttt{Animation::END\pagebreak{}}


\section{Example Program}

This small script will fade an RGB LED (the pins are defined by the
preprocessor macros RGBR, RGBG adnd RGBB and should be provided by
the calling program) to yellow, cause it to throb gently for a while,
then fade it to blue and back to yellow. Extra line breaks have been
added to aid readability:\\


\texttt{}
\begin{lstlisting}
Animation::FADE, RGBR, 64, 10, 
Animation::FADE, RGBG, 64, 10, 
Animation::FADE, RGBB, 0, 10, 
Animation::WAITEQ, 

Animation::REPEAT, 50, 

Animation::FADE, RGBR, 80, 10, 
Animation::FADE, RGBG, 80, 10, 
Animation::FADE, RGBB, 0, 10, 
Animation::WAITEQ, 

Animation::FADE, RGBR, 64, 10, 
Animation::FADE, RGBG, 64, 10, 
Animation::FADE, RGBB, 0, 10, 
Animation::WAITEQ, 

Animation::LOOP, 

Animation::FADE, RGBR, 0, 10, 
Animation::FADE, RGBG, 10, 10, 
Animation::FADE, RGBB, 64, 10, 
Animation::WAITEQ, 
Animation::DELAY, PAIR(2000), 

Animation::LOOP
\end{lstlisting}



\chapter{API}

The API contains both static and instance-based operations for controlling
animations.

A new animation sequence is created by calling the static createAnimation()
function and passing it a pointer to the program array. The animation
is started by calling it's start() function, stopped by calling its
stop() function, etc.

All animation processing is performed by a repeated call to the static
process() function.


\section{Static Functions}


\subsection{Animation {*}Animation::createAnimation({[}{[}const{]} uint8\_t {*}program{]});}

This function creates a new Animation object, optionally assigns the
program to it, and returns the new object to the user. The object
is also internally added to the animation processing list.

Animation {*}myAnimation = Animation::createAnimation(myProgram);


\subsection{void Animation::process();}

Iterate through the animation processing list and execute blocks of
instructions up to a delay class instruction for each animation registered.
This function needs to be called repeatedly (as fast as possible ideally)
from loop().

Animation::process();


\subsection{void Animation::addAnimation(Animation {*}anim);}

Add an animation constructed elsewhere to the internal processing
list. Under normal circumstances this function should not be used;
instead create the objects with the Animation::createAnimation() function.


\section{Instance-based functions}


\subsection{void setAnimation({[}const{]} uint8\_t {*}program);}

Replace (or set) the program for the current animation. If the animation
is running the new program is started from the beginning. Care should
be taken using this function as any LEDs will retain the state they
had at the time the program was replaced.

myAnimation->setAnimation(myOtherProgram);


\subsection{void start();}

Begin animating the animation. The program wil run, from the beginning,
either until it reaches an END instruction or the program is manually
terminated with stop().

myAnimation->start();


\subsection{void stop();}

Terminate a running animation. All LEDs retain their existing states.

myAnimation->stop();


\subsection{void nudge();}

Indicate to a program in a FOREVER loop that it should leave the loop
at the end of the current iteration.

myAnimation->nudge();

\pagebreak{}


\section{Example Program}

This small program executes the example animation from the previous
chapter.

\texttt{}
\begin{lstlisting}
#include <ULK.h>
#include <Animation.h>

#define RGBG 0
#define RGBB 1
#define RGBR 2

const uint8_t newLEDs[] = {
	Animation::FADE, RGBR, 64, 10,
	Animation::FADE, RGBG, 64, 10,
	Animation::FADE, RGBB, 0, 10,
	Animation::WAITEQ,
	Animation::REPEAT, 50,
	Animation::FADE, RGBR, 80, 10,
	Animation::FADE, RGBG, 80, 10,
	Animation::FADE, RGBB, 0, 10,
	Animation::WAITEQ,
	Animation::FADE, RGBR, 64, 10,
	Animation::FADE, RGBG, 64, 10,
	Animation::FADE, RGBB, 0, 10,
	Animation::WAITEQ,
	Animation::LOOP,
	Animation::FADE, RGBR, 0, 10,
	Animation::FADE, RGBG, 10, 10,
	Animation::FADE, RGBB, 64, 10,
	Animation::WAITEQ,
	Animation::DELAY, PAIR(2000),
	Animation::LOOP
};

void setup() {
	ULK.begin(1, 10);
	Animation *anim1 = Animation::createAnimation(newLEDs);
	anim1->start();
}

void loop() {
	Animation::process();
}
\end{lstlisting}

\end{document}
